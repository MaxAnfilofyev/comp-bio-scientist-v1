\documentclass{article} % For LaTeX2e
\usepackage{iclr2025,times}

% Optional math commands from https://github.com/goodfeli/dlbook_notation.
\input{math_commands.tex}

\usepackage{hyperref}
\usepackage{url}
\usepackage{graphicx}
\usepackage{subfigure}
\usepackage{booktabs}

% For theorems and such
\usepackage{amsmath}
\usepackage{amssymb}
\usepackage{mathtools}
\usepackage{amsthm}

% Custom
\usepackage{multirow}
\usepackage{color}
\usepackage{colortbl}
\usepackage[capitalize,noabbrev]{cleveref}
\usepackage{xspace}

%%%%%%%%%%%%%%%%%%%%%%%%%%%%%%%%
% THEOREMS
%%%%%%%%%%%%%%%%%%%%%%%%%%%%%%%%
\theoremstyle{plain}
\newtheorem{theorem}{Theorem}[section]
\newtheorem{proposition}[theorem]{Proposition}
\newtheorem{lemma}[theorem]{Lemma}
\newtheorem{corollary}[theorem]{Corollary}
\theoremstyle{definition}
\newtheorem{definition}[theorem]{Definition}
\newtheorem{assumption}[theorem]{Assumption}
\theoremstyle{remark}
\newtheorem{remark}[theorem]{Remark}

\graphicspath{{../figures/}} % To reference your generated figures, name the PNGs directly. DO NOT CHANGE THIS.

\begin{filecontents}{references.bib}
@book{goodfellow2016deep,
  title={Deep learning},
  author={Goodfellow, Ian and Bengio, Yoshua and Courville, Aaron and Bengio, Yoshua},
  volume={1},
  year={2016},
  publisher={MIT Press}
}
% Add biological references
@inproceedings{alphafold2021,
  title={Highly accurate protein structure prediction with AlphaFold},
  author={Jumper, John and Evans, Richard and Pritzel, Alexander and Green, Tim and Figurnov, Michael and Ronneberger, Olaf and Tunyasuvunakool, Kathryn and Bates, Russ and {\v{Z}}{\'i}dek, Augustin and Potapenko, Anna and others},
  booktitle={Nature},
  volume={596},
  number={7873},
  pages={583--589},
  year={2021},
  publisher={Nature Publishing Group}
}
\end{filecontents}

\title{
%%%%%%%%%TITLE%%%%%%%%%
TITLE HERE
%%%%%%%%%TITLE%%%%%%%%%
}

% Authors must not appear in the submitted version. They should be hidden
% as long as the \iclrfinalcopy macro remains commented out below.
% Non-anonymous submissions will be rejected without review.

\author{Anonymous}

\newcommand{\fix}{\marginpar{FIX}}
\newcommand{\new}{\marginpar{NEW}}

%\iclrfinalcopy % Uncomment for camera-ready version, but NOT for submission.

\begin{document}

\maketitle

\begin{abstract}
%%%%%%%%%ABSTRACT%%%%%%%%%
This paper presents a computational approach to understanding biological systems. Our method addresses key challenges in areas such as protein structure prediction, drug-target interactions, gene expression analysis, or systems biology modeling.

\textbf{Keywords:} computational biology, machine learning, biological modeling
%%%%%%%%%ABSTRACT%%%%%%%%%
\end{abstract}

\section{Introduction}
\label{sec:intro}
%%%%%%%%%INTRODUCTION%%%%%%%%%
Computational biology leverages machine learning and computational methods to understand biological systems and processes. This work explores novel approaches to [biological problem domain].

Key contributions include:
\begin{itemize}
    \item Novel computational method for [specific biological problem]
    \item Rigorous evaluation on relevant biological benchmarks
    \item Insights into underlying biological mechanisms
    \item Potential applications in drug discovery or biological research
\end{itemize}
%%%%%%%%%INTRODUCTION%%%%%%%%%

\section{Related Work}
\label{sec:related_work}
%%%%%%%%%RELATED WORK%%%%%%%%%
\subsection{Machine Learning in Computational Biology}

Deep learning has revolutionized many areas of computational biology including protein structure prediction \citep{alphafold2021}, drug discovery \citep{drug_discovery_2021}, and gene expression analysis.

\subsection{Methodological Approaches}

Previous work has explored various computational approaches for similar biological problems, demonstrating the value of integrating domain knowledge with modern machine learning techniques.

\subsection{Biological Benchmarks and Evaluation}

Standard biological benchmarks and evaluation protocols have been established for [relevant biological domain], providing standardized evaluation frameworks.
%%%%%%%%%RELATED WORK%%%%%%%%%

\section{Background and Biological Context}
\label{sec:background}
%%%%%%%%%BACKGROUND%%%%%%%%%
\subsection{Biological Problem Formulation}

[Specific biological problem and its scientific context]

\subsection{Computational Challenges}

Key computational challenges in [biological domain] include:
\begin{itemize}
    \item Complex biological relationships and interactions
    \item Limited availability of high-quality biological data
    \item Need for interpretability and biological plausibility
    \item Computational scalability for large biological datasets
\end{itemize}

\subsection{Existing Approaches}

Current approaches to [biological problem] typically rely on [describe standard approaches].
%%%%%%%%%BACKGROUND%%%%%%%%%

\section{Method}
\label{sec:method}
%%%%%%%%%METHOD%%%%%%%%%
\subsection{Problem Definition}

We formulate [biological problem] as [computational formulation].

\subsection{Model Architecture}

Our proposed method consists of [describe architecture components].

\subsubsection{Biological Constraints and Priors}

To ensure biological plausibility, we incorporate [domain-specific constraints].

\subsubsection{Training Strategy}

We employ [training approach] optimized for biological data characteristics.

\subsection{Implementation Details}

The method is implemented using [software stack]. We utilize biological databases such as [relevant databases] and leverage domain-specific preprocessing techniques.
%%%%%%%%%METHOD%%%%%%%%%

\section{Experimental Setup}
\label{sec:experimental_setup}
%%%%%%%%%EXPERIMENTAL SETUP%%%%%%%%%
\subsection{Datasets}

We evaluate our method on standard biological benchmarks including:
\begin{itemize}
    \item [Dataset 1]: [Description and biological relevance]
    \item [Dataset 2]: [Description and biological relevance]
\end{itemize}

\subsection{Baseline Methods}

We compare against established methods in computational biology:
\begin{itemize}
    \item [Baseline 1]: [Brief description]
    \item [Baseline 2]: [Brief description]
\end{itemize}

\subsection{Evaluation Metrics}

Following biological evaluation conventions, we use [primary metric] as our main evaluation criterion, supplemented by [secondary metrics].

\subsection{Experimental Protocol}

All experiments follow rigorous cross-validation procedures appropriate for biological data. Results are reported with statistical significance testing where applicable.
%%%%%%%%%EXPERIMENTAL SETUP%%%%%%%%%

\section{Results}
\label{sec:experiments}
%%%%%%%%%EXPERIMENTS%%%%%%%%%
\subsection{Main Results}

Our method achieves state-of-the-art performance on key biological benchmarks.

\begin{table}[ht]
\caption{Performance comparison on biological benchmarks}
\label{tab:main_results}
\begin{center}
\begin{tabular}{lccc}
\toprule
Method & Primary Metric & Secondary Metric & Biological Relevance \\
\midrule
Our Method & \textbf{[Best Value]} & [Value] & [Biological interpretation] \\
Baseline 1 & [Value] & [Value] & [Biological interpretation] \\
Baseline 2 & [Value] & [Value] & [Biological interpretation] \\
\bottomrule
\end{tabular}
\end{center}
\end{table}

\subsection{Analysis by Biological Subtype}

Performance varies across different biological subtypes, indicating that our method captures biologically relevant patterns.

\subsection{Ablation Studies}

Ablation experiments demonstrate the importance of incorporating biological domain knowledge.

% EXAMPLE FIGURE: REPLACE AND ADD YOUR OWN FIGURES / CAPTIONS
\begin{figure}[h!]
\centering
\includegraphics[width=0.8\textwidth]{example-image-a}
\caption{Biological analysis visualization. [Detailed description of biological insights]}
\label{fig:biological_analysis}
\end{figure}

\subsection{Statistical Validation}

We perform statistical tests to confirm the significance of our results, following biological data analysis best practices.
%%%%%%%%%EXPERIMENTS%%%%%%%%%

\section{Discussion}
\label{sec:discussion}
%%%%%%%%%DISCUSSION%%%%%%%%%
\subsection{Biological Insights}

Our model provides novel insights into [biological phenomena].

\subsection{Computational Advantages}

The method demonstrates improved computational efficiency and scalability for biological applications.

\subsection{Limitations and Future Work}

While promising, our approach has limitations including [describe limitations]. Future work will explore [potential extensions].

\subsection{Biological Implications}

This work contributes to the broader goal of understanding biological systems through computational methods.
%%%%%%%%%DISCUSSION%%%%%%%%%

\section{Conclusion}
\label{sec:conclusion}
%%%%%%%%%CONCLUSION%%%%%%%%%
We presented a novel computational approach to [biological problem] that demonstrates [key achievements]. Our method achieves [quantitative results] while providing [qualitative biological insights].

The integration of domain knowledge with modern deep learning techniques opens new avenues for computational biology research and applications in drug discovery and biological understanding.
%%%%%%%%%CONCLUSION%%%%%%%%%

\bibliography{references}
\bibliographystyle{iclr2025}

\appendix

\section*{\LARGE Supplementary Material}
\label{sec:appendix}

%%%%%%%%%APPENDIX%%%%%%%%%
\section{Additional Experimental Details}
\label{app:experimental_details}

Detailed experimental protocols, hyperparameter settings, and additional results are provided here.

\section{Dataset Descriptions}
\label{app:data_description}

Complete descriptions of all biological datasets used in this study.

\section{Software Implementation}
\label{app:implementation}

Full implementation details and software dependencies.

\section{Computational Resources}
\label{app:resources}

Information about computational resources and runtime requirements.

\section{Additional Figures and Tables}
\label{app:figures}

Supplementary visualizations and detailed results tables.
%%%%%%%%%APPENDIX%%%%%%%%%

\end{document}
