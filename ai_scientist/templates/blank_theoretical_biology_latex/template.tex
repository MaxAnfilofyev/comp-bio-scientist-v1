\documentclass{article} % For LaTeX2e
\usepackage{iclr2025,times}

% Optional math commands from https://github.com/goodfeli/dlbook_notation.
\input{math_commands.tex}

\usepackage{hyperref}
\usepackage{url}
\usepackage{graphicx}
\usepackage{subfigure}
\usepackage{booktabs}

% For theorems and such
\usepackage{amsmath}
\usepackage{amssymb}
\usepackage{mathtools}
\usepackage{amsthm}

% Custom
\usepackage{multirow}
\usepackage{color}
\usepackage{colortbl}
\usepackage[capitalize,noabbrev]{cleveref}
\usepackage{xspace}

%%%%%%%%%%%%%%%%%%%%%%%%%%%%%%%%
% THEOREMS
%%%%%%%%%%%%%%%%%%%%%%%%%%%%%%%%
\theoremstyle{plain}
\newtheorem{theorem}{Theorem}[section]
\newtheorem{proposition}[theorem]{Proposition}
\newtheorem{lemma}[theorem]{Lemma}
\newtheorem{corollary}[theorem]{Corollary}
\theoremstyle{definition}
\newtheorem{definition}[theorem]{Definition}
\newtheorem{assumption}[theorem]{Assumption}
\theoremstyle{remark}
\newtheorem{remark}[theorem]{Remark}

\graphicspath{{../figures/}} % To reference your generated figures, name the PNGs directly. DO NOT CHANGE THIS.

\begin{filecontents}{references.bib}
@book{goodfellow2016deep,
  title={Deep learning},
  author={Goodfellow, Ian and Bengio, Yoshua and Courville, Aaron and Bengio, Yoshua},
  volume={1},
  year={2016},
  publisher={MIT Press}
}
% Theoretical computational biology references
@inproceedings{smith1982evolution,
  title={Evolution and the theory of games},
  author={Smith, John Maynard},
  booktitle={Cambridge University Press},
  year={1982}
}
@inproceedings{nowak1994evolving,
  title={Spatial games and the evolution of cooperation},
  author={Nowak, Martin A and May, Robert M},
  booktitle={Proceedings of the National Academy of Sciences},
  volume={91},
  number={11},
  pages={4877--4881},
  year={1994},
  publisher={National Acad Sciences}
}
\end{filecontents}

\title{
%%%%%%%%%TITLE%%%%%%%%%
TITLE HERE
%%%%%%%%%TITLE%%%%%%%%%
}

% Authors must not appear in the submitted version. They should be hidden
% as long as the \iclrfinalcopy macro remains commented out below.
% Non-anonymous submissions will be rejected without review.

\author{Anonymous}

\newcommand{\fix}{\marginpar{FIX}}
\newcommand{\new}{\marginpar{NEW}}

%\iclrfinalcopy % Uncomment for camera-ready version, but NOT for submission.

\begin{document}

\maketitle

\begin{abstract}
%%%%%%%%%ABSTRACT%%%%%%%%%
This paper develops a mathematical model to investigate fundamental questions in theoretical computational biology. We formulate [biological question] and construct a computational framework that captures essential aspects of [biological system].

Using [mathematical approach, e.g., differential equations, game theory, stochastic processes, agent-based modeling], we explore how [key biological mechanisms] give rise to [observed biological phenomena]. Our simulations reveal [key findings about biological dynamics/evolution/stability].

This work provides theoretical insights into [biological domain] and generates testable predictions that can guide future experimental studies. The mathematical framework developed here establishes a foundation for understanding [broader biological implications].

\textbf{Keywords:} theoretical computational biology, mathematical modeling, evolutionary dynamics, game theory, systems biology
%%%%%%%%%ABSTRACT%%%%%%%%%
\end{abstract}

\section{Introduction}
\label{sec:intro}
%%%%%%%%%INTRODUCTION%%%%%%%%%
The field of theoretical computational biology bridges mathematical modeling with biological insight to address fundamental questions that are difficult or impossible to investigate through experimentation alone. This work develops a computational framework to understand [specific biological phenomenon or question].

\subsection{Biological Question}
What evolutionary and mechanistic principles allow [specific biological behavior/trait/system] to emerge and persist? This question lies at the intersection of [relevant biological subfields, e.g., evolutionary biology, systems biology, molecular biology].

\subsection{Computational Approach}
We employ [mathematical modeling framework, e.g., population genetics models, game theoretic approaches, dynamical systems, network theory] to capture the essential dynamics of [biological system]. Unlike purely descriptive models, our approach provides mechanistic explanations and generates testable predictions.

Key contributions:
\begin{itemize}
    \item Rigorous mathematical formulation of [biological question]
    \item Computational simulations revealing [key insights]
    \item Theoretical predictions amenable to experimental validation
    \item Framework extensible to related biological questions
\end{itemize}
%%%%%%%%%INTRODUCTION%%%%%%%%%

\section{Background and Theoretical Framework}
\label{sec:background}
%%%%%%%%%BACKGROUND%%%%%%%%%
\subsection{Biological Context}

Understanding [biological system/process] requires considering [key biological principles, constraints, evolutionary pressures]. The complexity arises from [describe key challenges in understanding this biological system].

\subsection{Mathematical Framework}

We model the system using [specific mathematical approach]:
\begin{itemize}
    \item \textbf{Variables}: [Define key variables quantifying biological states]
    \item \textbf{Parameters}: [Define parameters representing biological constraints/rates/interactions]
    \item \textbf{Dynamics}: [Specify how variables change over time/space]
    \item \textbf{Assumptions}: [Explicitly state simplifying assumptions and their justifications]
\end{itemize}

\subsection{Relationship to Existing Theory}

Our approach builds on [established theoretical frameworks] while addressing [specific gaps or novel aspects]. Previous work has [summarize key insights and limitations from related theoretical studies].
%%%%%%%%%BACKGROUND%%%%%%%%%

\section{Computational Model}
\label{sec:method}
%%%%%%%%%METHOD%%%%%%%%%
\subsection{Mathematical Formulation}

We formulate the biological system as follows:

\textbf{Governing Equations:}
\begin{align}
\frac{d x}{dt} &= f(x, y, \mathbf{p}) \label{eq:dynamics}\\
\frac{d y}{dt} &= g(x, y, \mathbf{p}) \label{eq:interactions}
\end{align}

where $x(t)$ and $y(t)$ represent [biological quantities, e.g., population frequencies, molecular concentrations], and $\mathbf{p}$ denotes the parameter vector.

\textbf{Equilibrium and Stability Analysis:}
At steady state, the system satisfies [equilibrium conditions]. Stability is determined by [eigenvalue analysis or other stability criteria].

\subsection{Computational Implementation}

The mathematical model is implemented computationally using [software/platform]. We employ [numerical methods, e.g., ODE solvers, Monte Carlo simulation, agent-based frameworks] to explore the model's behavior across parameter space.

\textbf{Simulation Protocol:}
\begin{enumerate}
    \item Initialize system with [initial conditions reflecting biological starting state]
    \item Evolve dynamics using [integration method] over [time/space scales]
    \item Record [quantities of interest for biological interpretation]
    \item Repeat across [parameter ranges relevant to biological scenarios]
\end{enumerate}

\subsection{Statistical Analysis Framework}

Following best practices for computational biology, we analyze simulation outputs using:
\begin{itemize}
    \item Sensitivity analysis to identify parameters controlling [biological outcome]
    \item Robustness analysis across [biological relevant parameter ranges]
    \item Statistical comparison of [different biological scenarios or hypotheses]
\end{itemize}
%%%%%%%%%METHOD%%%%%%%%%

\section{Computational Experiments}
\label{sec:experiments}
%%%%%%%%%EXPERIMENTS%%%%%%%%%
\subsection{Model Validation}

We validate the computational model against:
\begin{itemize}
    \item Analytical results for simplified cases
    \item Known biological behaviors from experimental literature
    \item Qualitative predictions of theoretical expectations
\end{itemize}

\subsection{Exploration of Parameter Space}

The computational framework enables systematic investigation of how [biological outcome] depends on [key parameters]. We perform extensive simulations across [parameter ranges informed by biological data/expectations].

\begin{figure}[h!]
\centering
\includegraphics[width=0.8\textwidth]{parameter_sweep}
\caption{Phase diagram showing [biological outcome] as a function of [key parameters]. Regions correspond to [different biological regimes].}
\label{fig:phase_diagram}
\end{figure}

\subsection{Key Findings}

Our simulations reveal:

\textbf{Finding 1:} [Specific quantitative result about biological behavior]
\begin{table}[ht]
\caption{Dependence of [biological quantity] on [parameter]}
\label{tab:key_results}
\begin{center}
\begin{tabular}{lccc}
\toprule
Parameter Value & Biological Outcome & Statistical Significance & Biological Interpretation \\
\midrule
Low & [Result] & [p-value] & [Interpretation] \\
Medium & [Result] & [p-value] & [Interpretation] \\
High & [Result] & [p-value] & [Interpretation] \\
\bottomrule
\end{tabular}
\end{center}
\end{table}

\textbf{Finding 2:} [Additional key insight]

\begin{figure}[h!]
\centering
\includegraphics[width=0.8\textwidth]{dynamics_plot}
\caption{Time evolution of [biological quantities] under [conditions]. The model predicts [biological insight].}
\label{fig:dynamics}
\end{figure}
%%%%%%%%%EXPERIMENTS%%%%%%%%%

\section{Biological Interpretation and Predictions}
\label{sec:interpretation}
%%%%%%%%%INTERPRETATION%%%%%%%%%
\subsection{Translation to Biology}

Our computational results provide insights into [biological mechanism/process]:

\textbf{Interpretation 1:} The observed [mathematical behavior] corresponds to [biological phenomenon], indicating that [biological mechanism] is sufficient to explain [observed biology].

\textbf{Interpretation 2:} The parameter dependence suggests that [biological constraint] plays a critical role in determining [biological outcome].

\subsection{Testable Predictions}

The mathematical framework generates specific predictions for experimental validation:

\begin{enumerate}
    \item \textbf{Prediction 1:} [Quantitative prediction about biological system behavior]
    \item \textbf{Prediction 2:} [Qualitative prediction about biological mechanism]
    \item \textbf{Prediction 3:} [Prediction about response to perturbations]
\end{enumerate}

\subsection{Comparison with Alternative Hypotheses}

Our model provides a [better/worse] explanation than [alternative models] because [specific advantages, e.g., mechanistic detail, quantitative accuracy, broader applicability].
%%%%%%%%%INTERPRETATION%%%%%%%%%

\section{Discussion}
\label{sec:discussion}
%%%%%%%%%DISCUSSION%%%%%%%%%
\subsection{Significance for Biology}

This work advances our theoretical understanding of [biological system/domain] by demonstrating how [mathematical principles] can give rise to [biological phenomena]. The computational approach reveals [insights that would be difficult to obtain experimentally].

\subsection{Relationship to Experiment}

Our theoretical predictions suggest new experimental directions in [biological subfield]. Specifically, testing [prediction] would validate or refute [aspect of the model], advancing both theoretical and experimental biology.

\subsection{Limitations}

The current model makes simplifying assumptions about [biological complexities not captured]. Future work should address [extensions to make model more biologically realistic].

\subsection{Future Directions}

The mathematical framework established here can be extended to:
\begin{itemize}
    \item Incorporate [additional biological details, e.g., spatial structure, molecular noise]
    \item Address related questions in [broader biological context]
    \item Guide experimental design for testing theoretical predictions
\end{itemize}
%%%%%%%%%DISCUSSION%%%%%%%%%

\section{Conclusion}
\label{sec:conclusion}
%%%%%%%%%CONCLUSION%%%%%%%%%
We developed a comprehensive mathematical model to investigate [biological question] through the lens of theoretical computational biology. Our computational approach successfully captured [key biological phenomena] and provided mechanistic insights into [biological mechanisms].

The framework generated specific quantitative predictions that can be experimentally tested, establishing a bridge between theoretical modeling and empirical biology. This work demonstrates the power of combining rigorous mathematics with computational exploration to advance our fundamental understanding of living systems.
%%%%%%%%%CONCLUSION%%%%%%%%%

\bibliography{references}
\bibliographystyle{iclr2025}

\appendix

\section*{\LARGE Supplementary Material}
\label{sec:appendix}

%%%%%%%%%APPENDIX%%%%%%%%%
\section{Mathematical Derivations}
\label{app:math}

Detailed derivation of [key equations/results].

\textbf{Equilibrium Analysis:}
[Mathematical analysis of steady states]

\textbf{Stability Conditions:}
[Linear stability analysis]

\textbf{Parameter Sensitivity:}
[Mathematical sensitivity analysis]

\section{Computational Details}
\label{app:computation}

\subsection{Implementation}
Code implementation using [language/libraries]. Available at [repository link if applicable].

\subsection{Hyperparameters and Numerical Methods}
\begin{itemize}
    \item Integration method: [ODE solver specifications]
    \item Time step: [numerical values]
    \item Convergence criteria: [stopping conditions]
    \item Random seeds: [for reproducibility]
\end{itemize}

\subsection{Computational Resources}
All simulations were performed on [hardware specifications] with total computation time of [estimated runtime].

\section{Additional Results}
\label{app:results}

Supplementary figures showing [additional parameter sweeps, robustness analyses, etc.].

\begin{figure}[h!]
\centering
\includegraphics[width=0.8\textwidth]{supp_figure1}
\caption{Additional analysis of [biological quantity] under [conditions].}
\label{fig:supp1}
\end{figure}

\section{Parameter Ranges and Biological Justification}
\label{app:parameters}

Detailed justification for parameter ranges used in simulations, based on [experimental data, theoretical expectations, biological literature].
%%%%%%%%%APPENDIX%%%%%%%%%

\end{document}
